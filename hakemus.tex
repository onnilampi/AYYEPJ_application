\documentclass[a4paper, 12pt, finnish]{report}
\usepackage[utf8]{inputenc}
\usepackage{amsfonts}
\usepackage{graphics}
\usepackage[finnish]{babel}
\usepackage{titlesec}
\titleformat{\chapter}
{\Large\bfseries}
{}            
{0pt}      
{\huge} 

\newcommand{\topic}{Hakemukseni edustajiston puheenjohtajaksi}
\usepackage{hyperref}
\hypersetup{pdfpagemode=UseNone, pdfstartview=FitH, colorlinks=true,urlcolor=red,linkcolor=blue,citecolor=black,pdftitle={\topic},pdfauthor={Onni Lampi}}
\setlength{\parindent}{0mm}
\setlength{\emergencystretch}{15pt}
\newcommand*{\findate}{\the\day.\the\month.\the\year}

\begin{document}



\includegraphics{onnilampi_ayy.jpg}
\section*{\topic}

Olen Onni, 25-vuotias 5. vuoden tietoliikennetekniikan opiskelija.
Olen kotoisin Helsingistä ja tässä muutaman opiskeluvuoden aikana AYY on tullut minulle hyvin läheiseksi ja rakkaaksi organisaatioksi.
Kuluvana vuonna olen AYYn hallituksessa päässyt suoraan ylioppilaskunnan ytimeen ja nälkä niin sanotusti kasvaa syödessä.
Siksi haenkin AYYn edustajiston puheenjohtajaksi, jotta voin edistää ylioppilaskunnan päätöksentekoa ja edustajiston toimintaa entistä konkreettisempaan suuntaan.\\

Persoonana olen rauhallinen, tunnollinen ja huumorintajuinen.
Vastuuni hoidan jämäkästi ja harkiten, en hätiköi asioita turhaan.
Pikaista reagointia vaativissa tilanteissa olen hyvä tekemään nopeita päätöksiä ja tässä auttaa aiemmin hankittu kokemus.\\

Minulle edustajisto on organisaationa tärkeä.
Ylioppilaskunnan kaikki ylimmät päätökset ja linjat tehdään edustajistossa, ja haluan olla olennainen osa tätä prosessia ensi vuonna.
Vuonna 2018 on edustajistossa strategiavuosi, jonka lisäksi vielä monet linjapaperit ja ohjesäännöt ovat tarkastelun alla.
Tämä tarkoittaa käytännössä sitä, että hallitus ja edustajisto tulevat käymään erittäin tiivistä dialogia läpi vuoden asioista, jotka eivät välttämättä edes mene maaliin silloisella hallituskaudella.
Tämän takia edustajiston puheenjohtajan ja varapuheenjohtajien rooli korostyy merkittävästi; pallo dialogin ja muun toiminnan suhteen ei yksinkertaisesti saa pudota!
 \\

Tänä vuonna edustajistossa on ensimmäistä kertaa koeistettu valiokuntia.
Valiokunnat ovat vuoden aikana osoittautuneet erittäin merkittäväksi tahoksi, ja haluaisin edustajiston puheenjohtajana jatkaa tällä tiellä.
Valiokuntien puheenjohtajien tulee olla erittäin tiiviissä yhteistyössä hallituksen ja muun toimiston kanssa; tällöin edustajiston edustavuus taataan entistä tehokkaammin!
Valiokuntiin on myös tärkeä valita kaikista edustajistoryhmistä nohevat henkilöt taustatahoistaan, jotta edustavuus ja mandaatti taataan.\\

Ei sovi myöskään unohtaa edustajistoryhmiä, joiden auttaminen ja tukeminen on ehdoton edellytys tehokkaalle toiminnalle.
Koen olennaiseksi, että toinen varapuheenjohtajista omistautuu lähes täysin edustajistoryhmien tukemiseen.
Toisen varapuheenjohtajan soisin vastaavan lähes täysin edustajiston kolmikielisyydestä, joka on edustajistossa jatkuva teema.
Mielestäni on tärkeää, ettei kolmikielisyyttä lakaista maton alle vaan se otetaan oikeasti tosissaan mahdollistavalla tasolla.
Valiokunnat pitäisin edustajiston puheenjohtajan kontolla siinä määrin, mihin sihteereinä toimivat hallituslaiset eivät taivu tai ehdi.\\


Terveisin, Onni

Espoossa \findate

\subsection*{Kysykää ihmeessä lisää!}
Onni Lampi\\
040 702 3841\\
omnez@telegram\\
contact@onnilampi.fi

\end{document}
